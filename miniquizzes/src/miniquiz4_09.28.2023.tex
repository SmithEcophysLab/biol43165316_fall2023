\documentclass[12pt, notitlepage]{article}   	% use "amsart" instead of "article" for AMSLaTeX format
\usepackage{geometry}                		% See geometry.pdf to learn the layout options. There are lots.
\geometry{a4paper}                   		% ... or a4paper or a5paper or ... 
%\geometry{landscape}                		% Activate for rotated page geometry
\usepackage[parfill]{parskip}    		% Activate to begin paragraphs with an empty line rather than an indent
\usepackage{graphicx}				% Use pdf, png, jpg, or eps§ with pdflatex; use eps in DVI mode
								% TeX will automatically convert eps --> pdf in pdflatex

\usepackage{hyperref}

%Use for images
\usepackage{graphicx}
\graphicspath{ {./images/} }

%SetFonts

\usepackage[T1]{fontenc}
\usepackage[utf8]{inputenc}

\usepackage{tgbonum}

%SetFonts

\title{
	\textbf{
		Mini-Quiz 4
	} \\
	\large BIOL 4301/6301 \\
	\large September 28, 2023 \\
}

\date{\vspace{-5ex}}

\def\wl{\par \vspace{\baselineskip}}

\begin{document}

{\fontfamily{phv}\selectfont %select helvetica (code = phv)

\large{Name:}

{\let\newpage\relax\maketitle}

\section{\small{Draw three figures to illustrate how (1) temperature, (2) litter C/N ratio,
and (3) soil microbial biomass influence litter decomposition. Next to each figure indicate
why you expect the trend you illustrated.}}

\newpage

\section{\small{How do you think eutrophication (the addition of nutrients to soils) will influence
soil organic carbon? Provide your answer in the form of a hypothesis with both the expected
response and biological reasoning.}}

\wl
\wl
\wl
\wl
\wl
\wl
\wl

\section{\small{How do you think elevated atmospheric CO2 will influence
soil organic carbon? Provide your answer in the form of a hypothesis with both the expected
response and biological reasoning.}}

\wl
\wl
\wl
\wl
\wl
\wl
\wl

\section{\small{How do you think increasing temperature will influence
soil organic carbon? Provide your answer in the form of a hypothesis with both the expected
response and biological reasoning.}}

\newpage

\section{\small{From the NEP equation (NEP = GPP - (Rplant + Rhet)), reductions in heterotrophic respiration would seem
to be a way to increase NEP and thus, reduce atmospheric CO2. What feedback may preclude this
from working? Provide your answer with text and a systems diagram.}}


} %end font selection

\end{document}