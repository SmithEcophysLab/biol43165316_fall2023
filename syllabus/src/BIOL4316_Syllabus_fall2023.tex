\documentclass[12pt, notitlepage]{article}   	% use "amsart" instead of "article" for AMSLaTeX format
\usepackage{geometry}                		% See geometry.pdf to learn the layout options. There are lots.
\geometry{a4paper}                   		% ... or a4paper or a5paper or ... 
%\geometry{landscape}                		% Activate for rotated page geometry
\usepackage[parfill]{parskip}    		% Activate to begin paragraphs with an empty line rather than an indent
\usepackage{graphicx}				% Use pdf, png, jpg, or eps§ with pdflatex; use eps in DVI mode
								% TeX will automatically convert eps --> pdf in pdflatex

\usepackage{hyperref}
		
%SetFonts

\usepackage[T1]{fontenc}
\usepackage[utf8]{inputenc}

\usepackage{tgbonum}

%SetFonts

\title{
	\textbf{
		BIOL 4301-029
	} \\
	\large Principles of Terrestrial Ecosystem Ecology \\
	\large Fall 2023
}

\date{\vspace{-5ex}}

\begin{document}

{\fontfamily{phv}\selectfont %select helvetica (code = phv)

\maketitle

\section{Course Description}
Students in this course will learn the fundamentals of ecosystem ecology.
This will include interactions between biological organisms and themselves as well as
their environment. Concepts taught will include water and energy cycling as well as carbon
and nutrient flows in natural and managed systems.
This will include aboveground and belowground processes.
As ecosystem ecology is the largest scale of ecology, the class will also cover necessary
concepts of individual, population, and community ecology.
Spatial extent of processes will extend to the globe. Temporal extent will extend to 
millenia. The class will consider applied aspects of global change and human decision making.
Students will be evaluated on their 
ability to discuss and disseminate course topics.

The course is part of a TTU-wide \textit{Creating Livable Futures} initiative.
The purpose of the initiative is to create opportunities for students to hear and
disseminate information related to global sustainability. More information on the
initiative is listed below.

\subsection{Class Time and Location}
Tuesdays and Thursdays 12:30-13:50

Science Building Room 204

\newpage

\subsection{Instructor}
Dr. Nick Smith \par
Experimental Sciences Building II (ESBII) Room 402D \par
806-834-7363 \par
nick.smith@ttu.edu \par
\textit{Meetings by appointment}

\subsection{Text}
Principles of Terrestrial Ecosystem Ecology (2nd Edition; 2011) 
by Chapin, Matson, and Vitousek \par
The book can be accessed from Springer here: 
\url{https://link.springer.com/book/10.1007/978-1-4419-9504-9}. Click on "Access this title on 
SpringerLink." It can also be accessed through the TTU library.

\section{Course Materials}
All course materials, including lecture slides, readings, activities, and code 
will be posted to a GitHub repository for the course.
The primary repository address is
\url{https://github.com/SmithEcophysLab/biol43165316_fall2023}.
The repository will include the syllabus, daily class notes, readings, and mini-quizzes.
The repository will also include other miscellaneous class materials as the semester
progresses. A README file will contain information on the repository, including
links to different sections at 
\url{https://github.com/SmithEcophysLab/biol43165316_fall2023/README.md}.

\section{Learning Objective}
This course will broadly focus on understanding the interactions between biological
organisms and their environment that drive cycles of energy, water, carbon, and nutrients
at local to global scales.
An emphasis will be placed on how these processes influence humanity in a changing world.
Applied concepts will consider human decisions as a means to reduce the negative impact
of global change. Class activities will be based on discussion and dissemination of ideas, 
including classic and recent scientific literature. 
Topics will be flexible and modified to match student interests where possible.
Note that broad course objectives meet the objectives of the \textit{Creating Livable Futures}
initiative (more information below).

\section{Attendance Policy}
Attendance will not be taken, but is strongly recommended. 
In class activity points will only be granted if students are in class.
Makeups will not be granted.

\section{Course Assessment}
\subsection{\textit{Participation and Engagement}}
Being an active and engaged participant in the class will benefit your understanding
of material as well as your peers'. Examples include asking questions, providing feedback,
and facilitating discussion.

\subsection{\textit{Mini-quizzes}}
Short “quizzes” will be given in class each week (typically on Thursdays). 
These quizzes will be used to stimulate discussion and to assess how well 
prior concepts were understood by the class. 
These will be graded for completion and participation in the ensuing class discussion.

\subsection{\textit{Reading feedback}}
Each week students will be required to read a a section of the book 
and produce a short summary as well as two questions that arose during their 
reading. 

\subsection{\textit{Recent literature feedback}}
Students not co-leading the current week’s 
discussion will be required to produce a summary and 
develop two questions based on each week’s chosen recent literature article.

\subsection{\textit{Literature review}}
The primary semester project will be to produce a literature review on a topic 
of the student's choice.
Broadly, the review should address a question or problem related 
to terrestrial ecosystem ecology and review the current state of knowledge on the topic.
The review should be forward thinking, in that it forms the
basis for understanding ecosystem dynamics moving forward.
The review should be novel in that it should not be similar to previously published
review papers.

Students will first develop a written proposal for their literature review. Dr. Smith will provide feedback. 
Students will then produce and present 
their review to the class at the end of the semester. 

This project will be done in groups of at least 3 people. Students are encouraged to receive help and guidance 
from the instructors as well as the class at large. 

The literature review will be assessed for completeness, breadth, originality, and presentation.
Students must have their project OKed by the instructor after the proposal and prior to
beginning the final project.

\newpage

\section{Grading}
Participation and Engagement: 15\% \par
Mini-quizzes: 15\% \par
Reading feedback: 10\% \par
Recent literature feedback: 10\% \par
Literature review idea proposal: 15\% \par
Final literature review presentation: 10\% \par
Final literature review: 25\% \par

Grades will be made available on Blackboard. 
All grades posted at the end of the course will be final, 
unless an error has been made in their calculation.
Please contact Dr. Smith if you feel your grade has been calculated incorrectly.

\section{Grading Scale}
A: $\geq$ 90\% \par
B: 80 – 90\% \par
C: 70 – 80\% \par
D: 60 – 70\% \par
F: $\leq$ 59.9\% \par

\section{Missing In-class Activities}
Students will be required to be in class to receive in-class activity points. 
Please contact Dr. Smith if you plan to miss class for a university function 
\textit{prior to class}. If class is missed due to an illness, 
please let Dr. Smith know as soon as possible (see COVID illness based absence policy below).

\subsection{Illness Based Absence Policy}
If at any time during this semester you feel ill, in the interest of your own health and 
safety as well as the health and safety of your instructors and classmates, you are 
encouraged not to attend face-to-face class meetings or events.  Please review the steps 
outlined below that you should follow to ensure your absence for illness will be excused. 
These steps also apply to not participating in synchronous online class meetings if you feel 
too ill to do so and missing specified assignment due dates in asynchronous online classes 
because of illness. If you are ill and think the symptoms might be COVID-19-related:
\begin{itemize}
	\item{Call Student Health Services at 806.743.2848 or your health care provider.  
	After hours and on weekends contact TTU COVID-19 Helpline at [TBA].}
	\item{Self-report as soon as possible using the Dean of Students COVID-19 webpage.
	This website has specific directions about how to upload documentation from a medical 
	provider and what will happen if your illness renders you unable to participate in 
	classes for more than one week.}
	\item{If your illness is determined to be COVID-19-related, all remaining 
	documentation and communication will be handled through the Office of the 
	Dean of Students, including notification of your instructors of the period of 
	time you may be absent from and may return to classes.}
	\item{If your illness is determined not to be COVID-19-related, please follow steps below.}
\end{itemize}

If you are ill and can attribute your symptoms to something other than COVID-19:
\begin{itemize}
	\item{If your illness renders you unable to attend face-to-face classes, participate 
	in synchronous online classes, or miss specified assignment due dates in asynchronous 
	online classes, you are encouraged to visit with either Student Health Services at 
	806.743.2848 or your health care provider.  Note that Student Health Services and 
	your own and other health care providers may arrange virtual visits.}
	\item{During the health provider visit, request a “return to school” note;}
	\item{E-mail the instructor a picture of that note;}
	\item{Return to class by the next class period after the date indicated on your note.}
\end{itemize}

\section{COVID-19 Statement}
The University will continue to monitor CDC, State, and TTU System guidelines concerning COVID-19. 
Any changes affecting class policies or temporary changes to delivery modality will be in 
accordance with those guidelines and announced as soon as possible. Students will not be 
required to purchase specialized technology to support a temporary course modality change, 
though students are expected to have access to a computer to access course content and 
course-specific messaging as needed. 

If you test positive for COVID-19, report your positive test through TTU's reporting system: 
\url{https://www.depts.ttu.edu/communications/emergency/coronavirus/}. Once you report a positive 
test, the portal will automatically generate a letter that you can distribute to your 
professors and instructors.

The TTU COVD-19 resource page is here: \url{https://www.depts.ttu.edu/communications/emergency/coronavirus/}.

\section{Special Considerations}
\subsection{ADA Statement}
Any student who, because of a disability, may require special arrangements in order to 
meet the course requirements should contact the instructor as soon as possible to make 
any necessary arrangements. Students should present appropriate verification from Student 
Disability Services during the instructor's office hours. Please note: instructors are 
not allowed to provide classroom accommodations to a student until appropriate verification 
from Student Disability Services has been provided. For additional information, please 
contact Student Disability Services in West Hall or call 806-742-2405.

\subsection{Religious Holy Days}
“Religious holy day” means a holy day observed by a religion whose places of worship 
are exempt from property taxation under Texas Tax Code §11.20. 
A student who intends to observe a religious holy day should make that intention known 
in writing to the instructor prior to the absence. 
A student who is absent from classes for the observance of a religious holy day shall be 
allowed to take an examination or complete an assignment scheduled for that day within a 
reasonable time after the absence. 
A student who is excused may not be penalized for the absence; however, the instructor 
may respond appropriately if the student fails to complete the assignment satisfactorily.

\section{TTU Resources for Discrimination, Harassment, and Sexual Violence}
Texas Tech University is committed to providing and strengthening an
educational, working, and living environment where students, faculty, staff,
and visitors are free from gender and/or sex discrimination of any kind.
Sexual assault, discrimination, harassment, and other \href{https://www.depts.ttu.edu/titleix/}{Title IX}
violations are not tolerated by the University. Report any incidents to the
Office for Student Rights & Resolution, (806)-742-SAFE (7233) or file a
report online at \url{titleix.ttu.edu/students}. Faculty and staff members at TTU
are committed to connecting you to resources on campus. Some of these
available resources are: TTU Student Counseling Center, 806- 742-3674,
\url{https://www.depts.ttu.edu/scc/} (Provides confidential support on
campus.) TTU 24-hour Crisis Helpline, 806-742-5555, (Assists students
who are experiencing a mental health or interpersonal violence crisis. If you
call the helpline, you will speak with a mental health counselor.) Voice of
Hope Lubbock Rape Crisis Center, 806-763-7273,
\url{voiceofhopelubbock.org} (24-hour hotline that provides support for
survivors of sexual violence.) The Risk, Intervention, Safety and Education
(RISE) Office, 806-742-2110, \url{https://www.depts.ttu.edu/rise/} (Provides a
range of resources and support options focused on prevention education
and student wellness.) Texas Tech Police Department, 806-742-3931,
\url{http://www.depts.ttu.edu/ttpd/} (To report criminal activity that occurs on
or near Texas Tech campus.)

\section{Student Support Statement}
The Office of Campus Access and Engagement works across Texas Tech
University to foster, affirm, engage, and strengthen all student
communities. For more information about services, opportunities for
participation, and ways in which Texas Tech can support your success in
college, please contact (806) 742-7025.

\section{Classroom Civility}
Texas Tech University is a community of faculty, students, and staff that
enjoys an expectation of cooperation, professionalism, and civility during
the conduct of all forms of university business, including the conduct of
student–student and student–faculty interactions in and out of the
classroom. Further, the classroom is a setting in which an exchange of ideas
and creative thinking should be encouraged and where intellectual growth
and development are fostered. Students who disrupt this classroom mission
by rude, sarcastic, threatening, abusive or obscene language and/or
behavior will be subject to appropriate sanctions according to university
policy. Likewise, faculty members are expected to maintain the highest
standards of professionalism in all interactions with all constituents of the
university (\url{www.depts.ttu.edu/ethics/matadorchallenge/ethicalprinciples.php}).

\section{Academic Integrity}
Academic integrity is taking responsibility for one's own class and/or course work, being
individually accountable, and demonstrating intellectual honesty and ethical behavior. Academic
integrity is a personal choice to abide by the standards of intellectual honesty and responsibility.
Because education is a shared effort to achieve learning through the exchange of ideas, students,
faculty, and staff have the collective responsibility to build mutual trust and respect. Ethical
behavior and independent thought are essential for the highest level of academic achievement,
which then must be measured. Academic achievement includes scholarship, teaching, and learning,
all of which are shared endeavors. Grades are a device used to quantify the successful
accumulation of knowledge through learning. Adhering to the standards of academic integrity
ensures grades are earned honestly. Academic integrity is the foundation upon which students,
faculty, and staff build their educational and professional careers. [Texas Tech University
(“University”) Quality Enhancement Plan, Academic Integrity Task Force, 2010].

\section{Plagiarism Statement}
Texas Tech University expects students to “understand the principles of academic integrity 
and abide by them in all class and/or course work at the University” (OP 34.12.5). 
Plagiarism is a form of academic misconduct that involves (1) the representation of words, 
ideas, illustrations, structure, computer code, other expression, or media of another as 
one's own and/or failing to properly cite direct, paraphrased, or summarized materials; 
or (2) self-plagiarism, which involves the submission of the same academic work more than 
once without the prior permission of the instructor and/or failure to correctly cite 
previous work written by the same student. Please review Section B of the TTU 
Student Handbook for more information related to other forms of academic misconduct, 
and contact your instructor if you have questions about plagiarism or other 
academic concerns in your courses. To learn more about the importance of 
academic integrity and practical tips for avoiding plagiarism, explore the 
resources provided by the TTU Library and the School of Law.

\section{Statement about Food Insecurity}
Any student who faces challenges securing their food or housing
and believes this may affect their performance in the course is
urged to contact the Dean of Students for support. Furthermore,
please notify the professor if you are comfortable in doing so. The
TTU Food Pantry is in Doak Hall room 117. Please visit the
website for hours of operation at \url{https://www.depts.ttu.edu/dos/foodpantry.php}.

\section{Creating Livable Futures}
This class is part of a campus-wide initiative called Creating Livable Futures, 
which is sponsored in part by the Texas Tech Center for Global Communication. 
As such, one of our objectives is to prepare you to communicate, 
in a fully interdisciplinary and global way, the challenges posed by pressing issues 
that speak to our collective wellbeing and sustainability. You will be asked to translate 
and communicate the work of leading thinkers on sustainability, and to expand discussing 
those materials through research experience and experiential learning.
These objectives will be met through discussion leads and the review paper. 

Your progress in communicating about global issues will be evaluated according to the 
Center for Global Communication rubric, so you will be invited to participate 
in one or more Creating Livable Futures activities outside of class that will 
complement class content. 
Planned Creating Livable Futures activities include participating in and attending 
speaker events and conferences, edit-a-thons, blogging and publication opportunities, 
student organizations, a book club, and even small scholarship opportunities for research. 

You’ll be informed of relevant opportunities and activities as they arise over 
the course of the semester.

\newpage

\section*{Schedule of Topics by Week}
Note: Book chapters in parentheses \par
08/21/2023 – \textbf{No class Tuesday}; Introductions, semester planning, and the Ecosystem Concept (Ch. 1) \par
08/28/2023 – Ecosystem Concept continued (Ch. 1) \par
09/04/2023 – \textbf{No class Thursday}; Climate and Soils (Ch. 2, 3) \par
09/11/2023 – Water and Energy Balance (Ch. 4) \par
09/18/2023 – Carbon inputs (Ch. 5) \par
09/25/2023 – Ecosystem carbon budgets (Ch. 7) \par
10/02/2023 – Fire w/ guest lecturer Dr. Dylan Schwilk \par
10/09/2023 – Plant Nutrient Use and Nutrient Cycling (Ch. 8, 9); \textbf{Literature review idea proposals due} \par
10/16/2023 – Species and Trophic Dynamics (Ch. 10, 11) \par
10/23/2023 – Temporal dynamics (Ch. 12) \par
10/30/2023 - \textbf{No class Thursday}; Landscape heterogeneity and ecosystem dynamics (Ch. 13) \par
11/06/2023 – Changes in the Earth System (Ch. 14) \par
11/13/2023 – Managing and sustaining ecosystems (Ch. 15) \par
11/20/2023 – \textbf{No classes} - Happy Thanksgiving! \par
11/27/2023 – \textbf{Literature review presentations} \par
12/05/2023 – \textbf{No class Thursday}; \textbf{Literature review presentations}; \textbf{Literature review due} \par

\section*{General Weekly Schedule}
Generally, each Tuesday will consist of a lecture by Dr. Smith followed by a discussion
of the reading. Students will turn in their reading
feedback at the end of Tuesday's lecture. Thursdays will generally begin with an in-class
mini-quiz and discussion. 
This will be followed by a discussion of a recent literature article and
(time permitting) an in-class activity.

} %end font selection

\end{document} 
